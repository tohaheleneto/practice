\documentclass[fontsize = 11pt,a4paper]{article}
\usepackage{fancybox,fancyhdr} 
\usepackage[utf8]{inputenc}
\usepackage[russian]{babel}
\usepackage{multicol}
\usepackage{titlesec}
\usepackage{amsmath}
\usepackage{indentfirst}




\setlength\parindent{12pt}
\usepackage[left=2cm,right=2cm,
    top=2cm,bottom=2cm]{geometry}
\title{\textbf{A Celestial-Mechanical Model for the Tidal Evolution\\
of the Earth-Moon System Treated as a Double Planet}}
\author{\textbf{A. A. Zlenko*}}
\date{\emph{Moscow Automobile and Roadway State Technical University, Moscow, Russia}\\
Received May 6, 2014; in final form, May 21, 2014}
\begin{document}
\maketitle

\thispagestyle{fancy}
\fancyhead[L]{\tiny \emph{ISSN 1063-7729, Astronomy Reports, 2015, Vol. 59, No. 1, pp. 72–87. \copyright Pleiades Publishing, Ltd., 2015.\\
Original Russian Text \copyright A.A. Zlenko, 2015, published in Astronomicheskii Zhurnal, 2015, Vol. 92, No. 1, pp. 80–96.\newline \newline}}


\fancyfoot[L] {*E-mail: zalaf121@mail.ru}
\renewcommand{\footrulewidth}{0.4pt}


\makeatletter
\def\headrule{{\if@fancyplain\let\headrulewidth\plainheadrulewidth\fi
\hrule\@height\headrulewidth\@width\headwidth
\vskip 2pt% 2pt between lines
\hrule\@height.5pt\@width\headwidth% lower line with .5pt line width
\vskip-\headrulewidth
\vskip-1.5pt}}
\makeatother


\indent\vbox{
\noindent\textbf{Abstract}—A celestial-mechanical model for the motion of two viscoelastic spheres in the gravitational
field of a massive point is considered, treating them as a double planet. The spheres move along quasicircular
orbits in a single plane, with their rotational axes perpendicular to this plane. The deformation of
the spheres is described using the classical theory of small deformations. A Kelvin–Voigt model is adopted
for the viscous forces. A system of evoutionary equations is obtained and applied to analyze the joint
translational–rotational tidal evolution of the Earth and Moon in the gravitational field of the Sun. This
system has been numerically integrated several billion years into the past and into the future. The results
are compared with the predictions of other theories, paleontological data, and astronomical observations.
\\
\\
\textbf{DOI}:10.1134/S1063772915010096
}
\begin{multicols}{2}

 \centerline{ 1. INTRODUCTION}
~\\
\indent The theory of tides originated with work by Newton
and Laplace. The main achievements in this
area were collected, systematized, and analyzed by
Darwin [1], and further developed by MacDonald [2],
who studied the evolution of the Earth–Moon system
without including the influence of the Sun. Goldreich
[3] used the method of MacDonald to investigate the
oblateness of the Earth and the influence of solar
tides, but neglecting the ellipticity of the lunar orbit,
and correctly averaged the equations of motion using
three time scales.\\
\indent The method of MacDonald was then used in various
other studies. Beletskii [4] investigated the tidal
evolution of the inclinations and rotations of celestial
bodies. Webb [5] studied the evolution of the
Earth–Moon system based on the ocean tides and
compared his results with the model of Goldreich [3].
Krasinsky [6] combined the methods of MacDonald
and Goldreich to reconstruct a dynamical history of
the Earth–Moon system. Touma andWisdom [7] developed
various models for tidal phenomena in detail.
It was shown that the evolution of the Earth–Moon
system based on the models of Darwin–Mignard and
Darwin–Cowley–Goldreich is essentially equivalent
to that predicted by the model of Goldreich.\\
\indent Efroimsky and Lainey [8] considered the effective
dissipation function Q, which is proportional to the
tidal frequency to the power $\alpha$. They studied the tidal
evolution of the Martian moon Phobos for $\alpha$ = 0.2, 0.3, 0.4. Note that $\alpha = 0$ in the model of MacDonald
and $\alpha$ = -1 in the model of Mignard [9, 10]. The
main distinguishing property of the approach proposed
by Ferraz-Mello et al. [11] is that, in contrast to
many studies based on the theory of Darwin, different
coefficients are introduced for the harmonics of the
tidal wave, instead of one Love number. A critical
analysis of the mathematical formulas in the above
theories describing the tidal moments, slowing of
planetary rotation, and the delay angle, as well as the
accuracy and range of applicability of the theories and
connections with rheological models, are considered
by Efroimsky and Williams [12] and Efroimsky and
Makarov [13]. Note that the qualitative conclusions
derived for the simpler MacDonald theory essentially
remain correct [12].\\
\indent The subsequent development of tidal theories is
concerned with the creation of rheological models.
Churkin [14–16] established a generalized theory of
the Love number and applied it to the rheological
models of Guk, Maxwell, Voigt, and others. His
theory for the rotation of the inelastic Earth was applied
to a Voigt model for the Earth’s interior, and
numerical estimates of rheological corrections to the
precession, nutation, and axial rotation of the Earth
were obtained. Efroimsky [17] introduced complex
Love numbers as a function of the tidal frequency to
study tides in the case of a rotational–orbital resonance
between a planet and one of its satellites.
Vil’ke [18] developed a method for separating motions
and averaging in systems with an infinite number
of degrees of freedom, aimed at studying the
motions of deformable bodies using a classical linear
\end{multicols}

\pagebreak

\begin{multicols}{2}
elasticity theory for small deformations and a Kelvin–
Voigt model for the viscous forces. This method
was used to investigate the evolution of the orbital
and rotational motions of a viscoelastic planet in a
central Newtonian force field [19, 20]. The model for a
celestial body of Markov and Minyaev [21] includes
an isotropic, viscoelastic layer and a rigid core. A
qualitative analysis of the motion of the moons of
Mars is given, and the model parameters were refined
based on the observations of the secular acceleration
of Phobos. Vil’ke and Shatina [22] studied the tidal
evolution of the motion of the Earth–Moon system in
the gravitational field of the Sun, treating the Moon
as a point mass.
Let us now turn to our model describing a double
planet [23–25].
\begin{center}
 2. MATHEMATICAL MODEL\\
FOR THE MOTION OF TWO VISCOELASTIC\\
SPHERES IN THE GRAVITATIONAL FIELD\\
OF A FIXED CENTRAL BODY
\end{center}
 \centerline{\emph{2.1. Formulation of the Problem}}
In the unperturbed motion, the barycenter C of the
two uniform rigid spheres $O_{1}$  and $O_{2}$ with masses m1
and m2 moves in a circular, Keplerian orbit in a fixed
plane in the gravitational field of a stationary massive
point mass M. The spheres $O_{1}$ and $O_{2}$, in turn,
move in circular Keplerian orbits about the barycenter
C in the plane of its motion. The spheres rotate
with specified constant angular speeds about axes
passing through their centers of mass perpendicular
to the plane of their orbital motion. All four motions
are independent of each other. This formulation of
the problem is possible because we have made the
assumptions

$m2 \ll m1 \ll M; r_{i0} \ll  R_{2} \ll R_{1}, $ \hfill(1)\\
where $r_{i0}$ (i = 1, 2) are the radii of the spheres, $R_{1}$
is the distance from the gravitating center to the
barycenter, and $R_{2}$ is the distance between the centers
of mass of $O_{1}$ and $O_{2}$ (we will further identify
the names of the spheres with their centers of mass).
These assumptions are satisfied, for example, by the
Sun–Earth–Moon system.\\
In the perturbed motion, we treat the spheres as
uniform, isotropic, viscoelastic bodies. Perturbations
arise due to the deformation of the bodies in response
to the centrifugal and gravitational forces. Since we
are studying evolutionary motions, we assume that
the centers of mass of the spheres move along quasicircular
orbits.\\
To describe the motion, we specify an inertial
coordinate frame OXY Z fixed to the gravitating
center O, with the spheres moving in the OXY
plane. We specify Koenig coordinate systems ${O_i}{X_i}{Y_i}$
with the points ${O_i}$ (i = 1, 2). The position of the barycenter C in the OXY Z system is specified by
the vector ${R_1} = \vec{OC} ({R_1} \cos {\lambda_1}, {R_1} \sin {\lambda_1}$, 0), where
$|{R_1}| = {R_1}$, and ${\lambda_1}$ is the angle between ${R_1}$ and the
OX axis. The position of ${O_2}$ relative to ${O_1}$ in the
${O_1}{X_1}{Y_1}{Z_1}$ frame is specified by the vector ${R_2} = {O_1}{O_2}({R_2} \cos {\lambda_2}, {R_2} \sin {\lambda_2}, 0)$, where 
$|{R_2}| = {R_2}$,  and
${\lambda_2}$ is the angle between ${R_2}$ and the $O{X_1}$ axis. The
deformed state of the bodies is described by the classical
theory of elasticity for small deformations. We
adopted a Kelvin–Voigt model for the viscous forces,
with the dissipation function ${D_i}[{\dot{u}_i}]$ proportional to
the elastic-force function ${W_i}[{\dot{u}_i}]$, with the coefficient
of proportionality ${X_i}$ (the viscosity coefficient):
${D_i}[{\dot{u}_i}] = {X_i}{W_i}[{\dot{u}_i}$], (2)
where ${u_i}({r_i}, t)$ is the shift in the points of the body ${O_i}$
due to the deformations, ${\dot{u}_i} = d{u_i}/dt$ (here and below,
a dot above a quantity denotes a time derivative), and
${r_i}$ is the radius vector of the points in a sphere relative
to the center ${O_i}$ in the undeformed state.
\\
\indent The rotating spheres are associated with their own
coordinate systems ${O_i}{X_{ii}}{Y_{ii}}{Z_{ii}}$, where the ${O_i}{Z_{ii}}$ axis
is perpendicular to the orbital plane (the ${O_i}{Z_i}$ and
${O_i}{Z_{ii}}$ axes coincide). The positions of the points in
the viscoelastic sphere ${O_i}$ in the OXY Z coordinate
system are determined by the vector field
${\zeta_i}({r_i},t) = \vec{O{O_i}} + {\Gamma_i}({\varphi_i})({r_i}+{u_i}({r_i},t))$,
where
$ {\Gamma_i}({\varphi_i}(t)) =
 \begin{pmatrix} 
\cos{\varphi_i} & -\sin{\varphi_i}  & 0\\
\sin{\varphi_i} & \cos {\varphi_i} & 0\\
0 & 0 & 1
\end{pmatrix}
.(4)$
Here,${\Gamma_i}$ is the orthogonal operator for the translation
from the Koenig coordinates ${O_i}{X_i}{Y_i}{Z_i}$ to the
coordinates ${O_i}{X_{ii}}{Y_{ii}}{Z_{ii}}$ and ${\varphi_i}$ is the rotation of the
${O_i}{X_{ii}}{Y_{ii}}{Z_{ii}}$ system about the ${O_i}{Z_{ii}}$ axis (${\varphi_i}$ is the
angle between the ${O_i}{X_i}$ and ${O_i}{X_{ii}}$ axes).
In order to uniquely determine the positions of the
centers of mass of the spheres ${O_i}$ in the ${O_i}{X_{ii}}{Y_{ii}}{Z_{ii}}$
coordinate systems as the spheres move by ${\zeta_i}({r_i}, t)$,
we imposed the following conditions (relations) on
this motion:\\
\indent
$\int\limits_{V_i} {u_i}d{v_i} = 0,$
$\int\limits_{V_i} curl{u_i}d{v_i} = 0$

$(d{v_i} = d{x_{ii}}d{y_{ii}}d{z_{ii}})$,
\\ where ${V_i} = \{|{r_i}| < {r_{i0}} \}$ is the region occupied by Oi
in the undeformed state.
\end{multicols}

\pagebreak
\begin{multicols}{2}
The functional for the kinetic energy of the system
has the form
$T = \frac{1}{2}m{\dot{R}^2_1} + \frac{m1m2}{2m}{\dot{R}^2_2} + \frac{1}{2}\sum_{i=1}^{2}{J_i}[{u_i}]{\dot{\varphi}_i}^2 + 2{G_i}{\dot{\varphi}_i}+{T_{0i}})$,
\\ where 
${J_i}[{u_i}] $= $\int\limits_{V_i} [{e_3}\times ({r_i} + {u_i})]^2 {\rho_i}d{v_i}$,
${G_i} = \int\limits_{V_i}[{e_3} \times ({r_i} + {u_i}),\dot{{u_i}}]{\rho_i}d{v_i}, $
${T_{0i}} = \int\limits_{V_i} (\dot{{u_i}}^2) {\rho_i} d {v_i},$
${e_3}$ is the unit vector for the $O_i Z_{ii}$ axis perpendicular
to the OXY plane, and ${\rho_i}$ is the density of the sphere
$O_i$.\\
The potential energy associated with the gravitational interactions is given by\\
$\Pi=\Pi_1 + \Pi_2 + \Pi_3$, \\
where $\Pi_1 = -f \int\limits_{V_i} \{(R_1 - m_2/m)R_2$ \\
$+\Gamma_1 (r_1 + u_1)]^2\}^{-1/2}\rho_1 d v_1$\\
is the energy associated with the interaction between
the gravitating center and the viscoelatic body $O_1$,
$\Pi_2 = -f \int\limits_{V_2} \{(R_1 + m_1/m)R_2$ \\
$+\Gamma_2 (r_2 + u_2)]^2\}^{-1/2}\rho_2 d v_2$\\
is the energy associated with the interaction between
the gravitating center and the viscoelatic body $O_2$,
$\Pi_3 = -f  \int\limits_{V_i}\int\limits_{V_2} \{(R_2 +\Gamma_2(r_2 + u_2)$ \\
$-\Gamma_1 (r_1 + u_1)]^2\}^{1/2}\rho_1\rho_2 d v_1d v_2$\\
is the energy associated with the deformable spheres
$O_1$ and $O_2$, f = GM, G is the gravitational constant,
and $m = m_1 + m_2$. Taking into account the condition
(1) and neglecting terms of order $(R_2/R_1)^3(m_2/m)^3$
and higher order in smallness in the potential energy,
and leaving only terms that are linear $u_i$, we obtain
$\Pi = fm/R_1 - G{m_1}{m_2}/R_2 + {\Pi _\rho}$ (8)\\
$\Pi_\rho = \sum\limits_{k=1}^{2} \sum\limits_{i=1}^{2} f_{ki}/ {R_k}^3\int\limits_{V_i}[{r_i}{u_i}$ (9) \\
$-3(\xi_{ki},r_i)(\xi_{ki},u_i)]\rho_idv_i$,\\
$f_{1i} = f$,   $f_{2i} = Gm_{3-i}$, \\
$\xi_{ki} = \tau_{ki}[\cos(\lambda_k - \phi_i),\sin(\lambda_k - \phi_i),0],$\\
$\tau_{21} = -1,$   $\tau_{ki} = 1$   $(k \neq 2, i \neq 1).$ \\
Following [22], we introduced canonical Poincar$\dot{e}$
variables $\lambda_k, \Lambda_k$ (k = 1, 2 to describe the motion of
the barycenter and centers of mass $O_i$:
$\Lambda_1 = m (fR_1)^{1/2}$,   $\Lambda_2 = m_r (f_0R_2)^{1/2}$ , (10)\\
where $m_r = m_1m_2/m, f_0 = Gm.$\\
To describe the rotational motion of the bodies, we\\
used the Andoyer canonical variables $\varphi_i, I_i $  (i = 1, 2):\\
$I_i = J_i[u_i]\dot{\varphi_i} + G_i$,    (11)\\
where $J_i[u_i]$ and $G_i$ is defined in (6). \\
The equation of motion was written in the form of
the Routh equations \\
$\dot{\Lambda_k} = -  \frac{\partial\Re }{\partial{\lambda_k}},$
$\dot{\lambda_k} = \frac{\partial\Re }{\partial{\Lambda_k}},$
$\dot{I_i} = -  \frac{\partial\Re }{\partial{\varphi_i}},$ (12) \\

$\dot{\varphi_i} = -  \frac{\partial\Re }{\partial{I_i}},$
$\frac{d}{dt} {\nabla_{\dot{u_i}}}\Re - {\nabla_{u_i}}\Re - {\nabla_{\dot{u_i}}}D_i = 0.$
Here, $\Re$ is the Routh function, which has the form\\
$\Re = - \frac{{f^2}{m^3}}{2{\Lambda_1}^2} - \frac{{f_0}^2{m_r}^3}{2{\Lambda_2}^2}$ \hfill (13)\\
$+  \sum\limits_{i=1}^{2} \{ \frac{{I_i}^2}{2A_i} -\frac{{I_i}^2}{2{A_i}^2}J_{i1}[u_i] $\\
$- \frac{I_i}{A_i}( e_3, \int\limits_{V_i} (r \times \dot{u_i}) dv) + W_i[u_i]\} + {\Pi_\rho}$\\
where $A_i$ = 0.4${m_i}{r^2_{i0}}$ is the moment of inertia of the
undeformed sphere $O_i$,\\
$J_{i1}[u_i] = 2  \int\limits_{V_i}[(r_i, u_i) - (e_3, r_i)(e_3, u_i)]\rho_idv_i,$(14)\\ 
and an expression for $\Pi_\rho$ is give by (9). \\
\indent The equations of motion admit the integral of the
angularmomentum. When the Routh function (13) is
written out in detail, it can be shown that the angular
variables $\lambda_k$ and $\varphi_i$ appear in this function in the
combination $\psi_{ki} = \lambda_k - \varphi_i$. It follows that \\
$\dot{I}_i = - \frac{\partial\Re}{\partial\varphi_i} = - \sum\limits_{k=1}^{2}\frac{\partial\Re}{\partial\psi_{ki}}\cdot
\frac{\partial\psi_{ki}}{\partial{\varphi_i}} = \sum\limits_{k=1}^{2} \frac{\partial\Re}{\partial\psi_{ki}}$, \hfill (15)\\
$\dot{\Lambda_k} = -  \frac{\partial\Re }{\partial{\lambda_k}} = - \sum\limits_{i=1}^{2} \frac{\partial\Re}{\partial\psi_{ki}} 
\cdot \frac{\partial\psi_{ki}}{\partial{\Lambda_k}} = - \sum\limits_{i=1}^{2}\frac{\partial\Re}{\partial\psi_{ki}},$\\
$\dot{I_1} + \dot{I_2} + \dot{\Lambda_1} + \dot{\Lambda_2}$\\
$ = \sum\limits_{i=1}^{2} \sum\limits_{k=1}^{2} \frac{\partial\Re}{\partial\psi_{ki}} -  \sum\limits_{k=1}^{2} \sum\limits_{i=1}^{2}
 \frac{\partial\Re}{\partial\psi_{ki}} = 0,$ \\
${I_1} + {I_2} + {\Lambda_1} + t{\Lambda_2} = K_0 = const.$\\
 \centerline{\emph{2.2.Finding the Displacements of Points\\
in the Bodies due to their Deformation}}
\indent The system (12) cannot be integrated in explicit
form, since this is a quite complex system of differential
equations. Therefore, we applied the method
for separating motions in systems with an infinite
number of degrees of freedom [18]. Since we assumed
the rigidity of the elastic spheres $O_i$ were high, we
introduced the small parameters $\varepsilon_i$, proportional to
the ratio of the squares of the angular velocity of
rotation of a sphere at the initial time and of the lowest
frequency for the intrinsic elastic vibrations of the
sphere. The displacements $u_i$ are small, and can be
represented as a series in powers of $\varepsilon_i$:\\
$u_i(r_i,t) = \varepsilon_i u_{i1}(r_i,t) +  \varepsilon_i^2 u_{i2}(r_i,t) + ...,$\hfill (16) \\
$\varepsilon_i = \rho_i r^2_{i0} \varphi^2_i(0) / E_i,$ \hfill (17) \\
where $E_i$ is the Young’s modulus for the body $O_i$.\\
If $\varepsilon_i  = 0$, then $u_i(r_i, t) = 0$, and the equations of
the unperturbed motion follow from (12):\\
$\dot{\Lambda_1} = \dot{\Lambda_2} = \dot{I_1} = \dot{I_2} = 0,$\\
$\dot{\lambda_1} = \omega_1$,   $\dot{\lambda_2} = \omega_2$  $\dot{\varphi_1} = \omega_3$  $\dot{\varphi_2} = \omega_4$\\
where \\
$\omega_1 = \frac{f^2m^3}{\Lambda^3_1}$,    $\omega_2 = \frac{f^2_0 m^3_r}{\Lambda^3_2}$ \hfill (19)\\
$\omega_3 = \frac{I_1}{A_1}$,  $\omega_4 = \frac{I_2}{A_2}$.\\
In this case, the center of mass of the two bodies C
moves along a circular orbit about the fixed center
O with a constant angular velocity $\omega_1$, the bodies
$O_i$ move along circular orbits about their center of
mass C with a constant angular velocity $\omega_2$, and the
bodies $O_1$ and $O_2$ rotate on their axes with constant
angular velocities $\omega_3$ and $\omega_4$, normal to their orbital
plane passing through their centers of mass. \\
\indent It can be shown that, after the intrinsic vibrations
of the viscoelastic spheres have died away, including
only the first term $\varepsilon_i u_{i1}$ in the expansion of $u_i(r_i, t)$ in
powers of $\varepsilon_i$ in (16), the last equations in (12) reduce
to the two relations\\
$\nabla_{u_i}W_i[\varepsilon_i u_{i1} + \chi_i \varepsilon_i \dot{u_{i1}}]$
$ = \rho_i \{ \omega^2_{2+i} [ r_i - (e_3,r_i)e_3]$\\
$+  \sum\limits_{k=1}^{2} (f_{ki} / R^3_k)[3(\xi_{ki},r_i) \xi_{ki} - r_i] \}$ (i = 1, 2),\\
where\\
$\nabla_{u}W_i[\varepsilon_i u] = \frac{\rho_ir^2_{i0}\dot{\varphi^2_i}(0)}{2(1 + v_i)}$\hfill (21)\\
$ \times (\frac{1}{1 - 2v_i}\nabla div u + \Delta u),$ \\
and $v_i$ is the Poisson coefficient for the matter in the
sphere $O_i$.\\
Equation (20) can be written in the form\\
${\varepsilon_i} \nabla_{u_i} W_i [u_{i1} + \chi_i \dot u_{i1}]$ \hfill (22) \\
$=\rho_i [\omega^2_{2+i}(2r_i/3 + B_o r_i) +  \sum\limits_{k=1}^{2} 3 (f_{ki}/R^3_k)B_{ki} r_{i}],$\\
where\\
$B_0 = 
\begin{pmatrix} 
1/3 & 0 &  0 \\
0 & 1/3 & 0 \\
0 & 0 & -2/3
\end{pmatrix}, $
\\
$B_{ki} = \frac{1}{6}
\begin{pmatrix} 
3cos 2 \psi_{ki} + 1  & 3 sin 2 \psi_{ki} &  0 \\
3 sin 2 \psi_{ki} & -3cos 2 \psi_{ki} + 1 & 0 \\
0 & 0 & -2
\end{pmatrix}.$\\
All the quantities in the right-hand side of (22) are
calculated for the unperturbed motion.\\
Taking into account the fact that the stresses on
the surfaces of the deformable bodies $O_i$ are zero (i.e.,
the boundary conditions for the functions $u_{i1}(r_i, t)$
have the form $\sigma_{in} = 0$), with accuracy to within firstorder
terms in the small quantity $\chi_i$, the solution of
(22) has the form\\
$u_i(r_i,t) \approx \varepsilon_i u_{i1} = u_{i11} + u_{i12} + u_{i13},$ \hfill (23) \\
$u_{i11} = \rho_i / E_i \omega^2_{2 + i} [-2/3(d_{i1} r^2_i + d_{i2} r^2_{i0})$\\
$+ a_{i1}(B_0 r_i, r_i) + (a_{i1} r^2_i + a_{i2} r^2_{i0})B_0]r_i,$\\
$u_{i12} = 3 \rho_i / E_i \sum\limits_{k=1}^{2} f_{ki} R^{-3}_k [a_{i1}(B_{ki}r_i,r_i)r_i$\\
$ + (a_{i2}r^2_i + a_{i3} r^2_{i0}) B_{ki}r_i],$\\
$u_{i13} = -3 \rho_i / E_i \sum\limits_{k=1}^{2} f_{ki} R^{-3}_k \chi_i (\omega_k - \omega_{2+i})$\\
$\times [a_{i1}(\frac{\partial B_{ki}}{\partial \psi_{ki}} r_i, r_i)r_i + (a_{i2}r^2_i +  a_{i3} r^2_{i0}) \frac{\partial B_{ki}}{\partial \psi_{ki}} r_i ], $\\
where\\
$d_{i1} = \frac{(1 + v_i)(1 - 2v_i)}{10(1 - v_i)},$\\
$d_{i2} =- \frac{(3 - v_i)(1 - 2v_i)}{10(1 - v_i)},$\\
$a_{i1} =- \frac{1+ v_i}{5 v_i + 7},$   $a_{i2} =- \frac{(1+ v_i)(2 + v_i)}{5 v_i + 7}. $\\
The structure of $u_i(r_i, t)$ is such that the first term
$u_{i11}$ describes axially symmetrical, elastic deformation
of the sphere $O_i$, which is compressed by the
action of the centrifugal force associated with the
rotation about the $O_iZ_{ii}$ axis passing through the
center of mass. The second term $u_{i12}$ characterizes
the deformation of the body $O_i$ due to the external
gravitational fields of the two other bodies. These
fields also give rise to gravitational tides, given by the
third term $u_{i13}$, which contains the viscosity coefficient
$\chi_i$ and influences the evolution of the motion.

\centerline{\emph{2.3. Simplification and Averaging\\
of the Equations of Motion}}
Let us write the canonical equations (12) in more
detail, taking into account the Routh function and the
displacements $u_i(r_i, t) \approx \varepsilon u_{i1}:$\\
$\dot\Lambda_k = 3  \sum\limits_{i=1}^{2} \rho_i f_{ki} R^{-3}_k$ \hfill (24) \\
$ \times \int\limits_{V_i}[(\frac{\partial{\xi_{ki}}}{\partial{\lambda_k}},ri)(\xi_{ki} \varepsilon_i u_{i1})$\\
$+ (\xi_{ki}, r_i) (\frac{\partial{\xi_{ki}}}{\partial{\lambda_k}}, \varepsilon_i u_{i1})] dv_i,$\\
$\dot\lambda_k  = \omega_k - 3 \sum\limits_{i=1}^{2} \rho_i f_{ki} R^{-4}_k \partial R_k / \partial \Lambda _k$ \hfill (25)\\
$ \times  \int\limits_{V_i} [ r_i \varepsilon_i u_{i1}) - 3(\xi_{ki}, r_i)(\xi_{ki},\varepsilon_i u_{i1})] d v_i,$ \\
$\dot I_i = 3 \rho_i \sum\limits_{k=1}^{2}  f_{ki} R^{-3}_k $ \hfill (26) \\
$ \times  \int\limits_{V_i} [(\frac{\partial{\xi_{ki}}}{\partial{\varphi_i}}, r_i)(\xi_{ki}, \varepsilon_i u_{i1})$ \\
$ + ( \xi_{ki}, r_i) (\frac{\partial{\xi_{ki}}}{\partial{\varphi_i}},  \varepsilon_i u_{i1})]  d v_i,$ \\
$\dot \varphi_i = \omega_{2 + i} - 2 {\rho_i} \frac{\omega_{2 + i}}{A_i}$ \hfill (27) \\
$\times \int\limits_{V_i}[ (r_i,  \varepsilon_i u_{i1}) - (e_3,r_i)(e_3, \varepsilon_i u_{i1})] d v_i$ \\
$- \frac{\rho_i}{A_i}[e_3,\int\limits_{V_i}(r \times \varepsilon_i  \dot u_{i1})d v_i].$\\
Subsituting the resulting variables (23) into the
equations of motion (24)–(27) and calculating the
necessary cumbersome integrals yields the system of
equations of motion\\
$\dot\Lambda_k  = -18 \sum\limits_{k=1}^{2} \rho ^2_i / E_i D_{i2} m_{3-i} \omega^2_k / m$ \hfill (28) \\
\end{multicols}

\thispagestyle{fancy}
\fancyhead[L]{TIDAL EVOLUTION OF THE EARTH–MOON SYSTEM}

\end{document}