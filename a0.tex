\documentclass[12pt]{article}
\usepackage{fancybox,fancyhdr} 
\usepackage[utf8]{inputenc}
\usepackage[russian]{babel}
\usepackage{multicol}
\usepackage{titlesec}
\parindent=1cm

\begin{document}

\thispagestyle{fancy}
\fancyhead[L]{\tiny ISSN 1063-7729, Astronomy Reports, 2015, Vol. 59, No. 1, pp. 72–87. \copyright Pleiades Publishing, Ltd., 2015.\\
Original Russian Text \copyright A.A. Zlenko, 2015, published in Astronomicheskii Zhurnal, 2015, Vol. 92, No. 1, pp. 80–96.\newline \newline}
\renewcommand{\headrulewidth}{1pt}

\fancyfoot[L] {*E-mail: zalaf121@mail.ru}
\renewcommand{\footrulewidth}{0.4pt}


\makeatletter
\def\headrule{{\if@fancyplain\let\headrulewidth\plainheadrulewidth\fi
\hrule\@height\headrulewidth\@width\headwidth
\vskip 2pt% 2pt between lines
\hrule\@height.5pt\@width\headwidth% lower line with .5pt line width
\vskip-\headrulewidth
\vskip-1.5pt}}
\makeatother

\textbf{A Celestial-Mechanical Model for the Tidal Evolution
of the Earth–Moon System Treated as a Double Planet}
\begin{center}
\textbf{A. A. Zlenko*}
\end{center}
\begin{center}
Abstract—A celestial-mechanical model for the motion of two viscoelastic spheres in the gravitational
field of a massive point is considered, treating them as a double planet. The spheres move along quasicircular
orbits in a single plane, with their rotational axes perpendicular to this plane. The deformation of
the spheres is described using the classical theory of small deformations. A Kelvin–Voigt model is adopted
for the viscous forces. A system of evoutionary equations is obtained and applied to analyze the joint
translational–rotational tidal evolution of the Earth and Moon in the gravitational field of the Sun. This
system has been numerically integrated several billion years into the past and into the future. The results
are compared with the predictions of other theories, paleontological data, and astronomical observations.
\end{center}

\textbf{DOI}:10.1134/S1063772915010096

\begin{multicols}{2}
\begin{center}
 1. INTRODUCTION
\end{center}
The theory of tides originated with work by Newton
and Laplace. The main achievements in this
area were collected, systematized, and analyzed by
Darwin [1], and further developed by MacDonald [2],
who studied the evolution of the Earth–Moon system
without including the influence of the Sun. Goldreich
[3] used the method of MacDonald to investigate the
oblateness of the Earth and the influence of solar
tides, but neglecting the ellipticity of the lunar orbit,
and correctly averaged the equations of motion using
three time scales.
The method of MacDonald was then used in various
other studies. Beletskii [4] investigated the tidal
evolution of the inclinations and rotations of celestial
bodies. Webb [5] studied the evolution of the
Earth–Moon system based on the ocean tides and
compared his results with the model of Goldreich [3].
Krasinsky [6] combined the methods of MacDonald
and Goldreich to reconstruct a dynamical history of
the Earth–Moon system. Touma andWisdom [7] developed
various models for tidal phenomena in detail.
It was shown that the evolution of the Earth–Moon
system based on the models of Darwin–Mignard and
Darwin–Cowley–Goldreich is essentially equivalent
to that predicted by the model of Goldreich.
Efroimsky and Lainey [8] considered the effective
dissipation function Q, which is proportional to the
tidal frequency to the power $\alpha$ . They studied the tidal
evolution of the Martian moon Phobos for $\alpha$ = 0.2,
0.3, 0.4. Note that $\alpha$ = 0 in the model of MacDonald
and $\alpha$ = -1 in the model of Mignard [9, 10]. The
main distinguishing property of the approach proposed
by Ferraz-Mello et al. [11] is that, in contrast to
many studies based on the theory of Darwin, different
coefficients are introduced for the harmonics of the
tidal wave, instead of one Love number. A critical
analysis of the mathematical formulas in the above
theories describing the tidal moments, slowing of
planetary rotation, and the delay angle, as well as the
accuracy and range of applicability of the theories and
connections with rheological models, are considered
by Efroimsky and Williams [12] and Efroimsky and
Makarov [13]. Note that the qualitative conclusions
derived for the simpler MacDonald theory essentially
remain correct [12].
The subsequent development of tidal theories is
concerned with the creation of rheological models.
Churkin [14–16] established a generalized theory of
the Love number and applied it to the rheological
models of Guk, Maxwell, Voigt, and others. His
theory for the rotation of the inelastic Earth was applied
to a Voigt model for the Earth’s interior, and
numerical estimates of rheological corrections to the
precession, nutation, and axial rotation of the Earth
were obtained. Efroimsky [17] introduced complex
Love numbers as a function of the tidal frequency to
study tides in the case of a rotational–orbital resonance
between a planet and one of its satellites.
Vil’ke [18] developed a method for separating motions
and averaging in systems with an infinite number
of degrees of freedom, aimed at studying the
motions of deformable bodies using a classical linear
\end{multicols}

\pagebreak

\begin{multicols}{2}
elasticity theory for small deformations and a Kelvin–
Voigt model for the viscous forces. This method
was used to investigate the evolution of the orbital
and rotational motions of a viscoelastic planet in a
central Newtonian force field [19, 20]. The model for a
celestial body of Markov and Minyaev [21] includes
an isotropic, viscoelastic layer and a rigid core. A
qualitative analysis of the motion of the moons of
Mars is given, and the model parameters were refined
based on the observations of the secular acceleration
of Phobos. Vil’ke and Shatina [22] studied the tidal
evolution of the motion of the Earth–Moon system in
the gravitational field of the Sun, treating the Moon
as a point mass.
Let us now turn to our model describing a double
planet [23–25].
\begin{center}
 2. MATHEMATICAL MODEL
FOR THE MOTION OF TWO VISCOELASTIC
SPHERES IN THE GRAVITATIONAL FIELD
OF A FIXED CENTRAL BODY
\end{center}
2.1. Formulation of the Problem\\
In the unperturbed motion, the barycenter C of the
two uniform rigid spheres $O_{1}$  and $O_{2}$ with masses m1
and m2 moves in a circular, Keplerian orbit in a fixed
plane in the gravitational field of a stationary massive
point mass M. The spheres $O_{1}$ and $O_{2}$, in turn,
move in circular Keplerian orbits about the barycenter
C in the plane of its motion. The spheres rotate
with specified constant angular speeds about axes
passing through their centers of mass perpendicular
to the plane of their orbital motion. All four motions
are independent of each other. This formulation of
the problem is possible because we have made the
assumptions

$m2 \ll m1 \ll M; r_{i0} \ll  R_{2} \ll R_{1}, (1) $

where $r_{i0}$ (i = 1, 2) are the radii of the spheres, $R_{1}$
is the distance from the gravitating center to the
barycenter, and $R_{2}$ is the distance between the centers
of mass of $O_{1}$ and $O_{2}$ (we will further identify
the names of the spheres with their centers of mass).
These assumptions are satisfied, for example, by the
Sun–Earth–Moon system.\\
In the perturbed motion, we treat the spheres as
uniform, isotropic, viscoelastic bodies. Perturbations
arise due to the deformation of the bodies in response
to the centrifugal and gravitational forces. Since we
are studying evolutionary motions, we assume that
the centers of mass of the spheres move along quasicircular
orbits.\\
To describe the motion, we specify an inertial
coordinate frame OXY Z fixed to the gravitating
center O, with the spheres moving in the OXY
plane. We specify Koenig coordinate systems ${O_i}{X_i}{Y_i}$
with the points ${O_i}$ (i = 1, 2). The position of the barycenter C in the OXY Z system is specified by
the vector ${R_1} = \vec{OC} ({R_1} \cos {\lambda_1}, {R_1} \sin {\lambda_1}$, 0), where
$|{R_1}| = {R_1}$, and ${\lambda_1}$ is the angle between ${R_1}$ and the
OX axis. The position of ${O_2}$ relative to ${O_1}$ in the
${O_1}{X_1}{Y_1}{Z_1}$ frame is specified by the vector ${R_2} = {O_1}{O_2}({R_2} \cos {\lambda_2}, {R_2} \sin {\lambda_2}, 0)$, where 
$|{R_2}| = {R_2}$,  and
${\lambda_2}$ is the angle between ${R_2}$ and the $O{X_1}$ axis. The
deformed state of the bodies is described by the classical
theory of elasticity for small deformations. We
adopted a Kelvin–Voigt model for the viscous forces,
with the dissipation function ${D_i}[{\dot{u}_i}]$ proportional to
the elastic-force function ${W_i}[{\dot{u}_i}]$, with the coefficient
of proportionality ${X_i}$ (the viscosity coefficient):
${D_i}[{\dot{u}_i}] = {X_i}{W_i}[{\dot{u}_i}$], (2)
where ${u_i}({r_i}, t)$ is the shift in the points of the body ${O_i}$
due to the deformations, ${\dot{u}_i} = d{u_i}/dt$ (here and below,
a dot above a quantity denotes a time derivative), and
${r_i}$ is the radius vector of the points in a sphere relative
to the center ${O_i}$ in the undeformed state.
//
The rotating spheres are associated with their own
coordinate systems ${O_i}{X_{ii}}{Y_{ii}}{Z_{ii}}$, where the ${O_i}{Z_{ii}}$ axis
is perpendicular to the orbital plane (the ${O_i}{Z_i}$ and
${O_i}{Z_{ii}}$ axes coincide). The positions of the points in
the viscoelastic sphere ${O_i}$ in the OXY Z coordinate
system are determined by the vector field

\end{multicols}

\thispagestyle{fancy}
\fancyhead[L]{TIDAL EVOLUTION OF THE EARTH–MOON SYSTEM}

\end{document}